\documentclass[a4paper, 10pt, french]{article}
% Préambule; packages qui peuvent être utiles
   \RequirePackage[T1]{fontenc}        % Ce package pourrit les pdf...
   \RequirePackage{babel,indentfirst}  % Pour les césures correctes,
                                       % et pour indenter au début de chaque paragraphe
   \RequirePackage[utf8]{inputenc}   % Pour pouvoir utiliser directement les accents
                                     % et autres caractères français
   % \RequirePackage{lmodern,tgpagella} % Police de caractères
   \textwidth 17cm \textheight 25cm \oddsidemargin -0.24cm % Définition taille de la page
   \evensidemargin -1.24cm \topskip 0cm \headheight -1.5cm % Définition des marges
   \RequirePackage{latexsym}                  % Symboles
   \RequirePackage{amsmath}                   % Symboles mathématiques
   \RequirePackage{tikz}   % Pour faire des schémas
   \RequirePackage{graphicx} % Pour inclure des images
   \RequirePackage{listings} % pour mettre des listings
% Fin Préambule; package qui peuvent être utiles

\title{Rapport de TP 4MMAOD : Génération d'ABR optimal}
\author{
ARGENTO Adrien Groupe 1 
\\ VINCENT Nicolas Groupe 1 
}

\begin{document}

\maketitle

%%%%%%%%%%%%%%%%%%%%%%%%%%%%%%%%%%%%%%%%%%%%%%
\section{Principe de notre  programme}
Pour ne pas subir le calcul des sommes pendant la construction de l'arbre, on choisit d'utiliser un tableau auxiliaire $sp$ de $n$ cases, stockant les sommes partielles des $p_i$. Autrement dit, chaque case $i$ de $sp$ contient les sommes de $j = 0$ à $i$ des $p_j$. Ce pré-travail coûte $O(n)$ opérations et $O(n)$ espace de stockage mémoire. Ainsi, notre somme se calcule en $O(1)$ opérations	 comme suit : $$\sum \limits_{l = i}^{j - 1} p_l =
	\left\{
		\begin{array}{ll}
		    sp[j - 1] - sp[i - 1] & \mbox{si } i \neq 0 \\
		    sp[j - 1] & \mbox{sinon.}
		\end{array}
	\right.$$

Sachant que les temps asymptotiques en itératif et récursif sont les mêmes, on choisit d'employer une méthode itérative, notamment pour éviter une surchage de pile en mémoire. \par\leavevmode\par

	Le principe de notre algorithme se présente comme suit : on définit deux tableaux auxiliaires, notés $c$ et $r$, de taille $\frac{n(n+1)}{2}$ pour stocker les coûts et racines des sous-arbres optimaux. Le calcul se fait en ascendant, en partant de la construction des plus petits sous-arbres pour construire les plus grands. La matrice des coûts -- présentée ci-dessous -- est triangulaire supérieure, contient sur sa diagonale pricipale les coûts des sous-arbres optimaux dont le nombre de sommets est égal à $1$. On rappelle que notre programme ne stocke que la partie triangulaire supérieure, on dispose d'un itérateur pour parcourir tous ces coefficients. \par\leavevmode\par
	\[
		\left(
		\begin{array}{ccccc}
		  c_{0,1} & c_{0,2} & \dots & \dots & c_{0,n-1} \\
		  &       c_{1,2}       &\dots & \dots  & c_{1,n-1} \\
		  &               & \ddots     &   & \vdots           \\
		  & \text{\huge0} &    &  \ddots   & \vdots       \\
		  &               &   &   & c_{n-2,n-1}
		\end{array}
		\right)
	\]  
	\par\leavevmode\par
	Parallèlement, on dispose d'une matrice $r$ construite de la même manière que $c$. L'idée de l'algorithme repose sur la construction pas à pas des diagonales de la matrice, débutant par la diagonale principale. Chaque diagonale représente des chaînes de sommets de longueur $j - i$ et dépend essentiellement des diagonales inférieures. Au fur et à mesure du calcul des coûts, on actualise $r$ en déterminant le sommet du sous-arbre optimal qui minimise le coût, tout en appliquant bien évidemment la restriction de Knuth. Ainsi, la profondeur moyenne de l'arbre complet est $c_{0,n-1}$, soit la diagonale la plus haute.
	Ainsi, on dispose de $r$ contenant l'intégralité des racines des sous-arbres minimisant le coût, on peut construire l'arbre en parcourant dans le sens inverse, de la diagonale la plus haute jusqu'à la diagonale principale.

	Nous proposerons en ouverture un algorithme plus performant en cache.
%%%%%%%%%%%%%%%%%%%%%%%%%%%%%%%%%%%%%%%%%%%%%%
\section{Analyse du coût théorique}
	Pour notre programme, comme on a $\frac{n(n-1)}{2}$ possibilités pour les couples $(i, j)$ avec $0 \leq i < j < n$, l'espace requis est en $\Theta(n^2)$.

	Par ailleurs, on sait que tout sous-arbre d'un ABR optimal est un ABR optimal. Ainsi, on choisit de prendre comme racine tous les noeuds de l'ABR en faisant varier $k$ de $i$ à $j - 1$. Lorsque l'on choisit que le $k$-ème noeud est la racine, on calcule récursivement le coût optimal de $i$ à $k - 1$ et de $k + 1$ à $j$. Comme l'on choisit $k$ parmi $j - i$ possibilités, la complexité temporelle est en $\Theta(n^3)$

	Une autre façon de voir notre problème est de considérer le total de $n^2$ sous-problèmes (ou du moins $\Theta(n^2)$). Sachant que chaque sous-problème est de complexité $\Theta(n)$ si l'on suppose que tous les sous-problèmes sont déjà résolus, on obtient une complexité totale en $\Theta(n^3)$.

	Maintenant, on peut restreindre le choix de $k$ en sachant que $r(i, j - 1) \leq r(i, j) \leq r(i + 1, j)$. 

	L'équation de Bellman devient $$C(i, j) = min_{r(i, j - 1) \leq k \leq r(i + 1, j)} C(i, k) + C(k + 1, j) + \sum \limits_{l = i}^{j - 1} p_l$$.

	Grâce à cette restriction, $\sum \limits_{i = 1}^{n} \sum \limits_{j = 1}^{n} r(i + 1, j) - r(i, j - 1) + 1 \leq n^2 + \sum \limits_{i = 1}^{n} r(i, n) + \sum \limits_{j = 1}^{n} r(n + 1, j) = O(n^2)$.

	Le temps de calcul devient $O(n^2)$.
  \subsection{Nombre  d'opérations en pire cas\,: }
	Le pire des cas se produit quand la profondeur de l'arbre est égal au nombre de sommets, c'est-à-dire quand on a un {\em unbalanced tree}. Le coût de recherche est en $O(n)$ et la structure de l'arbre se résume à une {\em linked list}. 
	Déterminons le nombre d'opérations en pire cas en prenant comme indicateurs le nombre de comparaisons, le nombre d'affectations, le nombre d'additions et le nombre de soutractions :
	\begin{enumerate}
		\item Initialisation de la diagonale principale : $2$ affectations $n$ fois.
		\item Boucle $1...n - 1$ : pas d'opérations.
		\item Boucle $0...n - d + 1$ : 6 affectactions (3 dans la boucle, 3 dans {\em get\_min}), 2 additions et 1 instruction conditionnelle.
		\item Boucle $r(i + 1, j, n)...r(i, j - 1, n)$ : 2 affectactions et 3 instructions conditionnelles.
	\end{enumerate}
	Le programme itératif contient la boucle $r(i + 1, j, n)...r(i, j - 1, n)$ imbriquée dans $0...n - d + 1$ imbriquée dans $1...n - 1$ correspondant à la somme :
	$$T(n) = \sum_{i=0}^{n - 1} 2 + \sum_{d=1}^{n - 1} \sum_{i = 0}^{n - d + 1} 9\sum_{k=r(i + 1, i + d, n)}^{r(i, i + d - 1, n)} 5 = 2n + \sum_{d=1}^{n - 1} \sum_{i = 0}^{n - d + 1} 45 (r(i, i + d - 1, n) - r(i + 1, i + d, n) + 1)$$ 
	$$\leq 2n + \sum_{d=1}^{n - 1} \sum_{i = 0}^{n - d + 1} 45n$$%\mbox{comme r(i, i + d - 1, n) - r(i + 1, i + d, n) + 1 \leq n}$$
	$$= 2n + \sum_{d=1}^{n - 1} (n - d + 2)n = 2n + n^2(n - 1) -n\frac{n(n - 1)}{2} + 2n(n - 1) = O(n^3)$$
	On en déduit un nombre d'opérations de $O(n^3)$
  \subsection{Place mémoire requise\,: }
Sachant que la construction de l'arbre complet depuis $r$ demande un espace mémoire de $2n$, notre algorithme utilise un espace mémoire de $2\frac{n(n+1)}{2} + n + 2n = O(n^2)$. \par\leavevmode\par 

  \subsection{Nombre de défauts de cache sur le modèle CO\,: }

Concernant le nombre de défauts de cache :
\begin{enumerate}
	\item Si le cache est assez grand pour contenir tous les espaces de stockage, c'est-à-dire si $Z \gg n(n+1) + 2n$ alors $Q(n, L, Z) = \frac{n(n+1) + 2n}{L} = O(\frac{n^2}{L})$. 
	\item Si le cache est trop petit pour contenir un espace de taille $n$, c'est-à-dire si $Z \ll n$ alors on a déjà $\frac{n}{L}$ défauts de cache pour écrire la diagonale principale.
\end{enumerate} 
    \paragraph{Justification\,: }


%%%%%%%%%%%%%%%%%%%%%%%%%%%%%%%%%%%%%%%%%%%%%%
\section{Compte rendu d'expérimentation (2 points)}
  \subsection{Conditions expérimentales}
     {\em Décrire les conditions permettant la reproductibilité des mesures: on demande la description
      de la machine et la méthode utilisée pour mesurer le temps.
     }

    \subsubsection{Description synthétique de la machine\,:} 
      {\em indiquer ici le  processeur et sa fréquence, la mémoire, le système d'exploitation. 
       Préciser aussi si la machine était monopolisée pour un test, ou notamment si 
       d'autres processus ou utilisateurs étaient en cours d'exécution. 
      } 

    \subsubsection{Méthode utilisée pour les mesures de temps\,: } 
      {\em préciser ici  comment les mesures de temps ont été effectuées (fonction appelée) et l'unité de temps; en particulier, 
       préciser comment les 5 exécutions pour chaque test ont été faites (par exemple si le même test est fait 5 fois de suite, ou si les tests sont alternés entre
       les mesures, ou exécutés en concurrence etc). 
      }

  \subsection{Mesures expérimentales}
    {\em Compléter le tableau suivant par les temps d'exécution mesurés pour chacun des 6 benchmarks imposés
              (temps minimum, maximum et moyen sur 5 exécutions)
    }

    \begin{figure}[h]
      \begin{center}
        \begin{tabular}{|l||r||r|r|r||}
          \hline
          \hline
            & coût         & temps     & temps   & temps \\
            & du patch     & min       & max     & moyen \\
          \hline
          \hline
            benchmark1 &      &     &     &     \\
          \hline
            benchmark2 &      &     &     &     \\
          \hline
            benchmark3 &      &     &     &     \\
          \hline
            benchmark4 &      &     &     &     \\
          \hline
            benchmark5 &      &     &     &     \\
          \hline
            benchmark6 &      &     &     &     \\
          \hline
          \hline
        \end{tabular}
        \caption{Mesures des temps minimum, maximum et moyen de 5 exécutions pour les 6 benchmarks.}
        \label{table-temps}
      \end{center}
    \end{figure}

\subsection{Analyse des résultats expérimentaux}
{\em Donner  une réponse justifiée  à la question\,: 
              les  temps mesurés correspondent ils  à votre analyse théorique (nombre d’opérations et défauts de cache) ?
}


\end{document}

%%%%%%%%%%%%%%%%%%%%%%%%%%%%%%%%%%%%%%%%%%%%%%
\section{Question bonus\,: programme {\tt mystere.c}(2 points)}
\subsection{Que fait le programme mystère et dans quel but? (1.5 point)}
{\em Dire brièvement ce que fait le programme mystere et quel est l'impact lors de l'exécution 
(i.e. lors de recherches avec le dictionnaire) de ce post-traitement.}

\subsection{Qu'en pensez-vous? (0.5 point) } 
{\em Répondre à l'argumentation en justifiant : soit que le programme mystere est (presque) optimal (justifier les hypothèses) ; soit qu'il n'est pas optimal en 
justifiant comment faire encore mieux.
}

%% Fin mise au format

